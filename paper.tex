\documentclass{article}
\usepackage{url}
\usepackage{mathtools}
\usepackage{tikz}
\usetikzlibrary{arrows,positioning}
\begin{document}
\title{Ripple Consensus Review}
\author{Peter Todd}
\date{FIXME}
\maketitle

\section{Activities performed}

The Ripple consensus review investigation had four major activities associated
with it:

\begin{enumerate}

    \item Reviewed whitepapers and development documentation at ripple.com

    \item Reviewed existing third-party critism.

    \item Setup and run a full node.

    \item Reviewed C++ implementation, specifically the 0.27.4 tag.\footnote{git commit 92812fe7239ffa3ba91649b2ece1e892b866ec2a}

\end{enumerate}


\section{Overall architecture}

Ripple has diverged significantly from the original concept of a decentralized
network representing money explicitly as debt relationship between
parties.\ref{btcmag-introducing-ripple} While the concept of a debt
relationship still exists in the form of trust lines between participants, the
bulk of the codebase and developer documentation now focuses\footnote{For
    instance the ``Tutorials'' section of the Ripple website only explains how
    to create transactions that modify the ledger; there is almost no information
available on how to actually use the trustlines feature.} on the use and
maintenance of a global ledger of transactions and account balances.
Additionally a native currency, XRP, has been added, which is used for
anti-spam transaction fees\footnote{A small amount of XRP is irrovocably
destroyed for every transaction.}, as well as to serve as a universal currency.


\subsection{Hashing and serialization}

Though various parts of the Ripple user API's and networking use industry
standard serialization formats like JSON and Google Protobuf, for
consensus-critical functionality Ripple uses a custom tag-length-value
serialization and hashing scheme. This is not unexpected as industry standard
serialization schemes rarely, if ever, account for the need to create hash
digests from represented objects.

\begin{equation}
    \textit{Serialize}(\text{obj}) = t_0 n_0 d_0 + \hdots + t_n n_n d_n
\end{equation}

Hashing of objects generally uses the following scheme, implemented by
STObject::getHash(), resulting in a $256\text{bit}$ digest:

\begin{equation}
    H(\text{obj}) = \text{SHA512}(p + \text{Serialize}(\text{obj}))[0:256\text{bits}]
\end{equation}

The prefix $p$ is per-object-type, guaranteeing that objects of different types
will always have a different hash.\footnote{This is commonly known as
\emph{tagged hashing} in the literature.} For objects containing signatures,
such as transactions, there is a similar but separate \emph{signing hash}
implemented by the function STObject::getSigningHash(). Unlike the standard
hash, the signing hash does not serialize signature fields.


\subsection{Transactions}

Unlike Bitcoin, transactions increment and decrement account balances; they do
not directly consume the output of other transactions. To prevent replay
attacks accounts and transaction have sequence numbers; a transaction is only
valid if the Sequence number is exactly one greater than the last-valided
transaction from the same account. Additionally an optional AccountTxnID field
is available for use within transactions; the transaction is only valid if the
sending account's previously-sent transaction matches the provided hash.

Authentication is performed with a simple signature scheme; a scripting system
is not available, nor is multisignature support. To implement the latter a
highly complex single-purpose scheme\ref{ripple-wiki-multisign} has been
proposed.


\subsection{Ledger}

Information on the exact structure of the Ripple ledger is somewhat spotty.

how exactly does the skiplist work?


\subsection{Consensus}


Claims from ripple: consensus protocol will maintain correctness so long as $f
\le (n-1)/5$ where $f$ is the number of Byzantian failures. However default UNL
list is just $5$ entries, thus $f \le (5-1)/5 = 0.8$ - no failures can be
tolerated.

Put antoher way, you only need to compromise $20\%$ of the nodes in the UNL to
censor a transactions/freeze funds. "failure to agree is agreement to defer"


Is there incentive to use anything but the default UNL? Remember that if you
choose a UNL more than $80\%$ different than others, you can be attacked on
that basis; your full node determines validity of transactions anyway. e.g. the
"add 1000 validators to UNL list" suggestion.

mixing structure: if tx A is rejected, dependent txs are all rejected; will
likely soon result in situation where you can't use different UNL


The Ripple consensus algorithm starts with a Unique Node List. 


what happens with UNL majority is not reached?

test: add extra entries to UNL list, see if ledgers close, if they don't think we know threshold


\section{Attacks}

reference c3 criteria?

big picture with attacks is there is not a minimum cost to them unlike PoW

evaluate attacks for denial-of-service and theft potential; evaluate minimum
cost to carry out attack; possible minimum manpower to carry out attack


Type of attack: DoS and/or theft

Minimum cost: $0, $1k, $10k, $100k

Scope: targetted, broad, global
Duration: hours, days, weeks, indefinite - How long will it take to fix the bug?

we don't attempt to put a dollar figure on harm done, as depends enormously on
how people use ripple


\subsection{Consensus Split}

1) exploit consensus failure.

DoS: Cost \$0, Scope: global, Duration: hours
Theft: Cost \$1k, Scope: varies, Duration: hours

not as synergistic with other attacks; obvious flaw that will be fixed;
time-limited attack


\subsection{Scalability}

DoS: Cost \$100k, Scope: global, Duration: weeks

simply produce a lot of txs, paid for via fees, either covertely or overtly.
System will fail, guaranteed.


\subsection{UNL: Jurisdictional}

DoS/Theft: Cost \$100k, Scope: varies, Duration: indefinite

centralization attack - legal attack against centralized UNL. note
jurisdictional issues with delegating consensus to UNL

biggest issue is how can ripple compete against systems without the global
consensus req? if I'm a bank in russia, why use ripple when it opens me to UNL
issues?


\subsection{Software backdoor}

DoS/Theft: Cost \$0, Scope: varies, Duration: varies

mention unsigned code here


\subsection{UNL: key theft}

prep: Cost \$0, Scope: global, Duration: N/A

Gain control of UNL signing keys through theft/compromise. 


\subsection{History simulation}

Theft: Cost \$0, Scope: targetted, Duration: weeks

from proof-of-stake terminology; simulated history is costless as no work
expended.

use backdoored software as example of how this can happen; also can happen with
jurisdictional attacks

if rippled sees two versions of history, what happens?

rewrite history attack - note lack of cost to rewrite because no PoW; single
implementation so likely we can hack all nodes at once

proof-of-work has a hard bound on rewrite attacks, because can't break laws of
physics

can talk about how cost to do a simulation attack is PoW $/targets; talk about slasher rule


\section{Synergistic Attacks}

do a directed graph of likely synergistic attack sequences


\bibliographystyle{plain}
\bibliography{paper}

\end{document}
